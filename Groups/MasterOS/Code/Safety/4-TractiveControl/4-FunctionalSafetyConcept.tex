\section{Functional safety concept}
% Reference: DS/ISO 26262-3 section 8
%
% Pre-requisites:
%    - Item definition
%    - Hazard analysis and risk assessment
%    - Safety goals
% 
% Reference: DS/ISO 26262-3 section 8.3.1

\subsection{Pre-existing information}
% The following information can be considered:
%    - Prior architectural assumptions (non-safety related)
%
% Reference: DS/ISO 26262-3 section 8.3.2

\subsection{Derivation of functional safety requirements}
% Functional safety requirements must be specified following DS/ISO 26262-8 section 6.
%
% A functional safety requirement specifies the safety measures and mechanisms that are
% required to comply with the safety goals.
% The functional safety concept is the set of functional safety requirements and the
% architectural elements of the item that they are assigned to.
% The functional safety concept is concerned with:
%    - Fault detection and failure mitigation
%    - Transitioning to a safe state after faults have been detected
%    - Fault tolerance mechanisms, where faults do not lead directly to violation of
%      the safety goals and a safe state is maintained (possibly with degradation)
%    - Fault detection and driver warning to reduce risk exposure time
%    - Arbitration logic to select "worst" fault in case of several simultaneous control
%      requests by different functions
%
% The functional safety requirements are derived from the safety goals and safe states,
% taking into account any prior architectural assumptions.
% One functional safety requirement can be valid for several safety goals.
%
% If a safe state cannot be reached by transition within an acceptable time interval,
% an emergency operation must be specified (e.g. if immediately switching off a system
% would not help, it might help to have it function in a degraded state).
%
% If assumptions are made about the reaction of the driver or other people at risk, the
% assumed actions must be specified in the functional safety concept and the controls
% available to the driver to do this must be specified.
%
% Required information:
%    - At least one functional safety requirement for each safety goal, specified by 
%      considering:
%        - Operating modes of the item
%        - Time interval of fault tolerance, i.e. how long before hazardous event occurs
%        - Safe states of the item (from safety goals)
%        - Emergency operation interval
%        - Functional redundancies (fault tolerance)
%
% Reference: DS/ISO 26262-3 section 8.4.2


\begin{description}
\item [4-FR1-C] To prevent violation of 4-SG1-A and 4-SG2-C, the blink light and the inactiv light should not be in the same state. If this happens the tractive system should shut down imediately.

\item [4-FR2-C] To prevent violation of 4-SG3-C a break system plausibility device should be installed, which shutdowns the tractive system if hard breaking is taking place while accelerating. Further more the speed pedal must have 2 sensors with different offset and/or slope. None of these may reach 100\% voltage or 0\% voltage. If one does not work the tractive system should shut down.

\item [4-FR3-C] To prevent violation of 4-SG4-A, breaks and steeringwheel should be mechanical and the breaking light circuit should be normally closed

\item [4-FR4-C] To prevent violation of 4-SG5-C the exposed parts of hte tractive system should be isolated from touching.

\item [4-FR5-C] To prevent violation of 4-SG6-D a mechanical switch for tractive system should be used to insure that no current can run through the tractive system.


\end{description}


\subsection{Allocation of functional safety requirements}
% Each functional safety requirement shall be allocated to the elements of the prior
% architectural assumptions.
%
% When allocating, the ASIL is inherited from the associated safety goal.
% If several functional safety requirements are allocated to the same architectural
% element, the highest ASIL is inherited unless it can be argued that the safety
% requirements are independent in the architecture.
% 
% If an item consists of more than one system, the functional safety requirements
% for the individual systems and their interfaces shall be specified.
%
% If functional safety requirements are implemented by external measures, the safety
% requirements that are implemented shall be derived and communicated.
% Additionally, the safety requirements of the interfaces with external measures
% shall be specified.
%
% Required information:
%    - Allocation of each functional safety requirement to elements of the architecture
%    - ASIL for development of each functional safety requirement
%    - Functional safety requirements of interfaces to external measures/internal systems
%
% Reference: DS/ISO 26262-3 section 8.4.3