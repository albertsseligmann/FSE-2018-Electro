\section{Safety goals}
% Reference: DS/ISO 26262-3 section 7.4.4.3 -- 7.4.4.6
%
% Pre-requisites:
%    - Item definition
%    - Hazard analysis and risk assessment
%
% The item is evaluated based ONLY on the item definition, i.e. WITHOUT considering
% internal safety mechanisms that are already implemented or are planned.
% Safety mechanisms that are intended to be implemented are instead part of the
% functional safety concept.
%
% Reference: DS/ISO 26262-3 section 7.4.1

\subsection{Pre-existing information}
% This section is for further supporting information such as
%    - Change impact analysis if applicable (only for existing items being changed)
%    - Relevant information on other independent items
%
% Reference: DS/ISO 26262-3 section 7.3.2

The cooling control system is a new item so there will be no change impact.
The motor controller is an EM-drive 500 which has a built in cut-off temperature.

\subsection{HAZARDOUS EVENT NAME}
% Every hazardous event with an ASIL from the hazard analysis shall have a safety goal
% defined to prevent the malfunction leading to the event.
% If several hazardous events cause very similar safety goals, the safety goals may be
% combined into a single safety goal.
%
% Safety goals are top-level safety requirements for the item. They are not expressed
% in terms of technological solutions but in terms of functional objectives cf. the
% item definition.
% The safety goals will lead to functional safety requirements required to avoid risk
% of hazardous events later on.
%
% Each safety goal shall be assigned the ASIL of the corresponding hazardous event.
% If multiple safety goals are combined, the highest ASIL must be assigned.
%
% If a safety goal can be achieved by transitioning to or maintaining one or more
% safe states, then those safe states shall be specified.
% A safe state can be e.g. that the item is turned off or that the vehicle is stationary.
%
% Required information:
%    - Safety goals for the hazardous event as described above
%    - For each safety goal:
%        - Safe states to transition to/maintain
%    - Each safety goal must be:
%        - Unambigous
%        - Comprehensible
%        - Atomic
%        - Internally consistent
%        - Feasible
%        - Verifiable
%        - Uniquely identifiable by some ID
%    - The set of safety goals must be:
%        - Complete
%        - Externally consistent
%        - Without duplication of information
%        - Maintainable
%
% References:
%    - Safety goals: DS/ISO 26262-3 section 7.4.4.3 -- 7.4.4.6
%    - Specification of safety requirements: DS/ISO 26262-8 section 6.4
%
% <<< Repeat this section for each hazardous event >>>

\subsection{2-OS9-HZ1 - ASIL C}
In this hazardous event the vehicle starts to free wheel while accelerating after a turn.
It is caused by the motorcontroller overheating.
To prevent this malfunction, we will set the following safety goal
\begin{description}
    \item[2-SG1-C] 
\end{description}

\subsection{2-OS3-HZ1 - ASIL A}
In this hazardous event the vehicle starts to free wheel while accelerating.
It is caused by the motorcontroller overheating.

\subsection{2-OS6-HZ1 - ASIL A}
In this hazardous event the vehicle starts to free wheel while accelerating before a turn.
It is caused by the motorcontroller overheating.

\subsection{2-OS7-HZ1 - ASIL A}
In this hazardous event the vehicle loses regenerative braking, causing reduction of the braking force.
It is caused by the motorcontroller overheating.
