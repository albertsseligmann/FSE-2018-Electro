\section{Technical requirements specification}
% Reference: DS/ISO 26262-4 section 6
%
% Pre-requisites:
%    - Functional safety concept
% 
% Reference: DS/ISO 26262-4 section 6.3.1

\subsection{Pre-existing information}
% The following information can be considered:
%    - Safety goals
%    - Functional concept
%    - Prior architectural assumptions
%
% Reference: DS/ISO 26262-4 section 6.3.2

\subsection{FUNCTIONAL SAFETY REQUIREMENT NAME}
% The technical safety requirements are a refinement of the functional safety
% requirements, taking in account the prior architectural assumptions, and
% turning the item-level functional requirements into system-level technical
% requirements.
%
% The technical safety requirements should be specified considering the following
% properties of the system:
%    - Functional safety requirements
%    - Prior architectural assumptions
%    - External interfaces, e.g. communication and user interfaces
%    - Environmental and functional constraints
%    - (Re)configuration requirements
%    - Dependencies on other systems/elements
%    - Requirements during production, operation, maintenance and repair
%
% If a system should implement functions that are not described by the technical
% safety requirements, a reference should be made to their specifqication.
%
% The technical safety requirements shall specify the response of the system to
% events that affect the achievement of safety goals, including:
%    - Detection, indication and control of faults in the system and communication
%    - Detection, indication and control of faults in external devices/systems
%    - Measures to maintain or transition to a safe state
%    - Measures to implement the warning and degradation concept
%    - Measures to prevent latent (undetected root cause, "random'') faults, e.g. self
%      tests on power up (only for ASILs C and D).
%
% Required information:
%    - Technical safety requirements for the functional safety requirement (see above)
%    - Transitions, fault tolerant time interval, emergency operation interval and
%      measures to maintain state for each safety mechanism that enables a safe state
%    - For safety requirements assigned ASILs C or D, safety mechanisms to prevent
%      latent faults
%
% References:
%    - DS/ISO 26262-4 section 6.1 -- 6.2 and 6.4.1 -- 6.4.5
%
% <<< Repeat this section for each hazardous event >>>
