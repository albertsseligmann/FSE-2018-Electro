\section{Integration testing report}
% Reference: DS/ISO 26262-4 section 8
%
% Pre-requisites:
%    - Safety goals
%    - Functional safety concept
%    - Technical safety concept
%    - System design specification
%
% Reference: DS/ISO 26262-4 section 8.3.1

\subsection{Pre-existing information}
% The following information can be considered:
%    - Vehicle architecture
%    - Technical safety concepts of other vehicle systems
%
% Reference: DS/ISO 26262-4 section 8.3.2

\subsection{Hardware-software integration testing}
% The developed hardware and software shall be integrated.
% The use of production-intent hardware and software is preferred, but modified
% hardware and software may be used where necessary for particular tests.
%
% The correct implementation of the technical safety requirements at the
% hardware-software level shall be demonstrated using the following methods,
% depending on the assigned ASIL:
%    - Requirements-based test against functional and non-functional requirements
%    - Fault injection test (via special test interface or special hardware)
%    - Back-to-back comparison test with model (if possible)
%
% The functional performance and timing of the safety mechanisms at the
% hardware-software level shall be demonstrated using the following methods,
% depending on the assigned ASIL:
%    - Back-to-back comparison test with model (if possible)
%    - Performance test of timing, power output etc. on intended hardware
%
% The consistent and correct implementation of the external and internal interfaces
% at the hardware-software level shall be demonstrated using the following methods,
% depending on the assigned ASIL:
%    - Test of external interfaces
%    - Test of internal interfaces
%    - Interface consistency check
%
% The effectiveness of the hardware fault detection mechanisms' diagnostic coverage
% shall be demonstrated using the following methods, depending on the assigned ASIL:
%    - Fault injection test (via special test interface or special hardware)
%    - Error guessing test (i.e. guessing what errors may happen and testing them)
%
% The level of robustness of the elements at the hardware-software level shall be
% demonstrated using the following methods, depending on the assigned ASIL:
%    - Resource usage test (either static or dynamic)
%    - Stress test (e.g. extreme temperatures, mechanical shocks, overloads, etc.)
%
% Required information:
%    - Results of technical safety requirements implementation tests
%    - Results of safety mechanism tests
%    - Results of interface tests
%    - Results of safety mechanism coverage tests
%    - Results of robustness tests
%
% Reference:
%    - Technical safety requirement implementation tests: DS/ISO 26262-4 Table 5
%    - Safety mechanism tests: DS/ISO 26262-4 Table 6
%    - Interface tests: DS/ISO 26262-4 Table 7
%    - Safety mechanism coverage tests: DS/ISO 26262-4 Table 8
%    - Robustness tests: DS/ISO 26262-4 Table 9

\subsection{System integration testing}
% The individual elements in the system shall be integrated in accordance with the
% system design.
%
% The correct implementation of the functional and technical requirements at the 
% system level shall be demonstrated using the following methods, depending on
% the assigned ASIL:
%    - Requirements-based test against functional and non-functional requirements
%    - Fault injection test (via special test interface or special hardware)
%    - Back-to-back comparison test with model (if possible)
%
% The functional performance and timing of the safety mechanisms at the
% system level shall be demonstrated using the following methods, depending on
% the assigned ASIL:
%    - Back-to-back comparison test with model (if possible)
%    - Performance test of whole system response times, actuator strength, etc.
%
% The consistent and correct implementation of the external and internal interfaces
% at the system level shall be demonstrated using the following methods, depending
% on the assigned ASIL:
%    - Test of external interfaces
%    - Test of internal interfaces
%    - Interface consistency check
%    - Test of interaction/communication between systems and system elements
%
% The effectiveness of the safety mechanisms' failure coverage shall be demonstrated
% using the following methods, depending on the assigned ASIL:
%    - Fault injection test (via special test interface or special elements)
%    - Error guessing test (i.e. guessing what errors may happen and testing them)
%    - Test derived from field experience
%
% The level of robustness at the system level shall be demonstrated using the
% following methods, depending on the assigned ASIL:
%    - Resource usage test (typically dynamic e.g. in a prototype or lab setup)
%    - Stress test (e.g. extreme temperatures, extreme user inputs, high loads, etc.)
%    - Test for interference resistance and robustness (e.g. EMC, EMI, and ESD tests)
%
% Required information:
%    - Results of requirements implementation tests
%    - Results of safety mechanism tests
%    - Results of interface tests
%    - Results of safety mechanism coverage tests
%    - Results of robustness tests
%
% Reference:
%    - Requirement implementation tests: DS/ISO 26262-4 Table 10
%    - Safety mechanism tests: DS/ISO 26262-4 Table 11
%    - Interface tests: DS/ISO 26262-4 Table 12
%    - Safety mechanism coverage tests: DS/ISO 26262-4 Table 13
%    - Robustness tests: DS/ISO 26262-4 Table 14

\subsection{Vehicle integration and testing}
% Each item shall be integrated into the vehicle and the integration shall be 
% tested.
%
% The correct implementation of the functional and technical requirements at the 
% vehicle level shall be demonstrated using the following methods:
%    - Requirements-based test against functional and non-functional requirements
%    - Fault injection test (via special test interface or special hardware)
%    - Long-term test
%    - User test under real-life conditions
%
% The functional performance and timing of the safety mechanisms at the
% vehicle level shall be demonstrated using the following methods, depending on
% the assigned ASIL:
%    - Performance test of whole system response times, actuator strength, etc.
%    - Long-term test
%    - User test under real-life conditions
%
% The consistent and correct implementation of the external interfaces at the system
% level shall be demonstrated using the following methods, depending on the
% assigned ASIL:
%    - Test of external interfaces
%    - Test of interaction/communication between systems during runtime
%
% The effectiveness of the safety mechanisms' failure coverage shall be demonstrated
% at the vehicle level, using the following methods, depending on the assigned ASIL:
%    - Fault injection test (via special test interface or special elements)
%    - Error guessing test (i.e. guessing what errors may happen and testing them)
%    - Test derived from field experience
%
% The level of robustness at the vehicle level shall be demonstrated using the
% following methods, depending on the assigned ASIL:
%    - Resource usage test (typically dynamic e.g. in a prototype or lab setup)
%    - Stress test (e.g. extreme temperatures, extreme user inputs, high loads, etc.)
%    - Test for interference resistance and robustness (e.g. EMC, EMI, and ESD tests)
%    - Long-term test
%
% Required information:
%    - Results of requirements implementation tests
%    - Results of safety mechanism tests
%    - Results of external interface tests
%    - Results of safety mechanism coverage tests
%    - Results of robustness tests
%
% Reference:
%    - Requirement implementation tests: DS/ISO 26262-4 Table 15
%    - Safety mechanism tests: DS/ISO 26262-4 Table 16
%    - Interface tests: DS/ISO 26262-4 Table 17
%    - Safety mechanism coverage tests: DS/ISO 26262-4 Table 18
%    - Robustness tests: DS/ISO 26262-4 Table 19