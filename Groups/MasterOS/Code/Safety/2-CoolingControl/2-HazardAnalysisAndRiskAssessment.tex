\section{Hazard analysis and risk assessment}
% Reference: DS/ISO 26262-3 section 7.4.1.1 -- 7.4.4.2
%
% Pre-requisites:
%    - Item definition
%
% The item is evaluated based ONLY on the item definition, i.e. WITHOUT considering
% internal safety mechanisms that are already implemented or are planned.
% Safety mechanisms that are intended to be implemented are instead part of the
% functional safety concept.
%
% Reference: DS/ISO 26262-3 section 7.4.1

\subsection{Pre-existing information}
% This section is for further supporting information such as
%    - Change impact analysis if applicable (only for existing items being changed)
%    - Relevant information on other independent items
%
% Reference: DS/ISO 26262-3 section 7.3.2

The cooling control system is a new item so there will be no change impact.
The motor controller is an EM-drive 500 which has a built in cut-off temperature.

\subsection{Situation analysis}
% The operational situations and operating modes in which an item's malfunctioning
% behaviour will result in a hazardous event are described.
% Consider both situations when the vehicle is correctly used AND when it is
% incorrectly used in a foreseeable manner.
% It is NOT necessary to consider situations in which the item is not expected to
% behave safely, i.e. when driving off-road.
%
% Required information:
%   - List of situations/modes in which the item may malfunction in a hazardous way
%
% Reference: DS/ISO 26262-3 section 7.4.2.1

The following situations will be considered:
\begin{description}
\item [2-OS1] Standstill - tractive system active
\item [2-OS2] Driving straight - constant speed
\item [2-OS3] Driving straight - accelerating
\item [2-OS4] Driving straight - decelerating
\item [2-OS5] Entering corner - constant speed
\item [2-OS6] Entering corner - accelerating
\item [2-OS7] Entering corner - decelerating
\item [2-OS8] Exiting corner - constant speed
\item [2-OS9] Exiting corner - accelerating
\item [2-OS10] Exiting corner - decelerating
\end{description}

We will not consider the following situtations: Standstill - tractive system inactive,
vehicle being pushed and vehicle undergoing maintenance/repairs.
In these situtations the cooling system should be inactive.
If the cooling system is active and should not be this should not pose any danger.

\subsection{Hazard identification}
% Identify hazards systematically using e.g. brainstorming, FMEA and tests.
%
% A hazard is a potential source of harm caused by malfunctioning behaviour.
% A hazard is defined in terms of conditions or behaviour that can be observed at
% the vehicle level assuming every other independent item/system is functioning
% correctly.
% A hazard can have many potential causes related to the implementation of the item,
% but these do NOT need to be considered in this hazard analysis and risk assessment,
% which is derived from a functional behaviour of the item.

% A hazardous event is a combination of a hazard and an operational situation.
% A hazardous event may have several simultaneous consequences, which must be
% considered together.
%
% Required information:
%    - List of hazards that can be observed at the vehicle level
%    - List of hazardous events for combinations of hazards and operational situations
%    - The consequences of each hazardous event
%
% Reference: DS/ISO 26262-3 section 7.4.2.2
% Definitions: DS/ISO 26262-1 term 1.57 -- 1.59

The following hazard will be considered:
\begin{description}
\item [2-HZ1] unexpected loss of acceleration and regenerative braking while tractive system is active - caused by EM-drive temperature cutoff
\end{description}

The operational situations and the hazards result in the following possible hazardous events:
\begin{center}
\begin{tabular}{l|l|l|l}
Operational situation	& Hazard	 & Hazardous event	& Consequences \\
2-OS1			& 2-HZ1	& 2-OS1-HZ1		& No danger, though bystanders may think driver is about to accelerate when he is not. \\
2-OS2			& 2-HZ1	& 2-OS2-HZ1		& No danger, though vehicle will decelerate slightly and may confuse the driver behind.
2-OS3			& 2-HZ1	& 2-OS3-HZ1		& No danger, though vehicle will decelerate slightly and may confuse the driver behind.
2-OS4			& 2-HZ1	& 2-OS4-HZ1		& Maybe dangerous, as braking force will suddenly decrease causing confusion.
2-OS5			& 2-HZ1	& 2-OS5-HZ1		& No danger
2-OS6			& 2-HZ1	& 2-OS6-HZ1		& No danger, though vehicle will decelerate slightly and may confuse the driver behind.
2-OS7			& 2-HZ1	& 2-OS7-HZ1		& Maybe dangerous, as braking force will suddenly decrease causing confusion.
2-OS8			& 2-HZ1	& 2-OS8-HZ1		& No danger, though vehicle will decelerate slightly and may confuse the driver behind.
2-OS9			& 2-HZ1	& 2-OS9-HZ1		& Maybe dangerous, as driver behind will expect vehicle to accelerate.
2-OS10			& 2-HZ1	& 2-OS10-HZ1   	& Maybe dangerous, as braking force will suddenly decrease causing confusion.




\subsection{Classification of hazardous events}
% All hazardous events shall be classified, unless they are outside the scope of
% DS/ISO 26262, e.g.  electric shock or fires not directly caused by malfunctioning
% E/E systems.
%
% If there is doubt about the classification of a hazard, it should be classified
% conservatively, i.e. a higher ASIL classification shall be given rather than a lower.
%
% Required information for each hazardous event:
%    - Estimate of severity class of potential harm from the event
%    - Probability class of exposure of each operational situation causing the event
%    - Estimate of controllability class of the event
%    - An ASIL for the event using the above parameters, ensuring that the operational
%      situations are not too granular so the exposure classes are inappropriately low
%
% References:
%	DS/ISO 26262-3 section 7.4.3 -- 7.4.4.2
%    - Severity class: DS/ISO 26262-3 Table 1
%    - Exposure probability class: DS/ISO 26262-3 Table 2
%    - Controllability class: DS/ISO 26262-3 Table 3
%    - ASIL determination: DS/ISO 26262-3 Table 4








\begin{center}
\begin{tabular}{l|l|l|l|l}
Hazardous event	& Severity	& Exposure probability	& Controllability	& ASIL	\\
2-OS1-HZ1		& S0		& E3					& C0				& N/A	\\
2-OS2-HZ1		& S1		& E2					& C1				& QM	\\
2-OS3-HZ1		& S2		& E4					& C1				& A	\\
2-OS4-HZ1		& S2		& E2					& C1				& QM	\\
2-OS5-HZ1		& S1		& E3					& C1				& QM	\\
2-OS6-HZ1		& S3		& E3					& C1				& A	\\
2-OS7-HZ1		& S2		& E4					& C1				& A	\\
2-OS8-HZ1		& S2		& E3					& C1				& QM		\\
2-OS9-HZ1		& S3		& E4					& C2				& C	\\
2-OS10-HZ1		& S1		& E3					& C1				& QM		\\

\end{tabular}
\end{center}
