\section{Technical requirements specification}
% Reference: DS/ISO 26262-4 section 6
%
% Pre-requisites:
%    - Functional safety concept
% 
% Reference: DS/ISO 26262-4 section 6.3.1

\subsection{Pre-existing information}
% The following information can be considered:
%    - Safety goals
%    - Functional concept
%    - Prior architectural assumptions
%
% Reference: DS/ISO 26262-4 section 6.3.2
The brake monitoring system architecturally consists of the following parts:
\begin{itemize}
\item Brake pedal position sensor
\item Computing system
\item Brake light.
\end{itemize}

% The technical safety requirements are a refinement of the functional safety
% requirements, taking in account the prior architectural assumptions, and
% turning the item-level functional requirements into system-level technical
% requirements.
%
% The technical safety requirements should be specified considering the following
% properties of the system:
%    - Functional safety requirements
%    - Prior architectural assumptions
%    - External interfaces, e.g. communication and user interfaces
%    - Environmental and functional constraints
%    - (Re)configuration requirements
%    - Dependencies on other systems/elements
%    - Requirements during production, operation, maintenance and repair
%
% If a system should implement functions that are not described by the technical
% safety requirements, a reference should be made to their specifqication.
%
% The technical safety requirements shall specify the response of the system to
% events that affect the achievement of safety goals, including:
%    - Detection, indication and control of faults in the system and communication
%    - Detection, indication and control of faults in external devices/systems
%    - Measures to maintain or transition to a safe state
%    - Measures to implement the warning and degradation concept
%    - Measures to prevent latent (undetected root cause, "random'') faults, e.g. self
%      tests on power up (only for ASILs C and D).
%
% Required information:
%    - Technical safety requirements for the functional safety requirement (see above)
%    - Transitions, fault tolerant time interval, emergency operation interval and
%      measures to maintain state for each safety mechanism that enables a safe state
%    - For safety requirements assigned ASILs C or D, safety mechanisms to prevent
%      latent faults
%
% References:
%    - DS/ISO 26262-4 section 6.1 -- 6.2 and 6.4.1 -- 6.4.5
%
% <<< Repeat this section for each hazardous event >>>
\subsection{1-FR1-D}
This functional safety requirement must be implemented by the brake light system.
This functional safety requirement prescribes that the brake light system must contain two different light emitting devices.

\paragraph{1-TR1-D}
The brake light must be implemented by two electrically parallel light emitting devices, so that the failure of one device does not affect the other.

\subsection{1-FR2-D}
This functional safety requirement must be implemented by the brake light system.
This functional safety requirement prescribes that the brake light control system must be a normally closed system.

\paragraph{1-TR2-D}
The brake light control system must supply power to the brake light when no control signal is detected (i.e. when the control signal is ``floating'').

\subsection{1-FR4-D}
This functional safety requirement shall be implemented by the brake pedal position sensor.
This functional safety requirement prescribes that the brake pedal position sensor shall measure the  brake pedal using two different position sensors, and that the tractive system must be disabled and the driver alerted if the sensors disagree.

\paragraph{1-TR3-D}
The brake pedal position sensor must be implemented by two different position sensors with differing transfer functions.

\paragraph{1-TR4-D}
The two position sensors mentioned in 1-TR3-D must be compared using non-programmable hardware to detect disagreements.
The comparison hardware must be adjustable to compensate for temperature and wear.

\paragraph{1-TR5-D}
If a disagreement between the two position sensors mentioned in 1-TR3-D is detected, a signal must immediately be sent to High Voltage Control to disable the tractive system.

\paragraph{1-TR6-D}
If a disagreement between the two position sensors mentioned in 1-TR3-D is detected, a designated light on the dashboard must be lit to warn the driver.
