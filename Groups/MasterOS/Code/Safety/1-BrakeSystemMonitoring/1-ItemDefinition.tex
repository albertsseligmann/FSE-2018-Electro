\section{Item definition}
% Reference: DS/ISO 26262-3 section 5

\subsection{Pre-existing information}
% This section is for further supporting information such as a product idea,
% a project sketch, relevant patents, the results of pre-trials, the documentation
% from predecessor items, relevant information on other independent items.
% Reference: DS/ISO 26262-3 section 5.3.2
The brake system consists of a pedal connected to a piston controlling the hydraulically activated brake discs on the wheels.
Additionally, the motors may be used for regenerative braking under some conditions.

\subsection{Functional requirements}
% A functional requirement is a requirement for what the item is supposed
% to accomplish.
% In general, functional requirements can be expressed in the form:
%     The item must do <requirement>
% Functional requirements are best specified by input/output listings,
% a mathemathical model/function or similar behaviour descriptions.
%
% Required information:
%   - Purpose of the item
%   - Functionality of the item, including operating modes and states of the item
%   - Assumptions on expected behaviour of the item
%
% Reference: DS/ISO 26262-3 section 5.4.1
The purpose of monitoring the brake system is to control the brake light and the regenerative braking system.
The brake system monitoring receives the position of the brake pedal and the speed of the vehicle as input.

The brake light has two states: shining and non-shining.
Whenever the brake pedal is pressed, the brake light must transition to the shining state.
Whenever the brake pedal is not pressed, the brake light must transitition to the non-shining state.

The brake system monitoring must always operate.

\subsection{Non-functional requirements}
% A non-functional requirement is a requirement for how the item should be.
% In general, non-functional requirements can be expressed in the form:
%     The item must be <requirement>
% Non-functional requirements are overall properties of the item such as quality
% requirements and constraints.
%
% Required information:
%    - Operational constraints such as prescribed behaviours for implementation
%    - FSE rule requirements
%
% Recommended information:
%    - Environmental constraints such as temperature and humidity
%    - Behaviour achieved by similar items, if any
%
% Reference: DS/ISO 26262-3 section 5.4.1
The brake pedal monitoring system must be resistant to the environmental conditions inside the cockpit.

\subsection{Boundary and interfaces}
% Required information:
%    - Elements of the item
%    - Assumptions about item's behaviour on other items
%    - Interactions with other items
%    - Functionality required by other items
%    - Functionality required from other items
%
% Recommended information:
%    - Distribution of functions among the involved systems
%    - Operating scenarios which impact the functionality of the item
%
% Reference: DS/ISO 26262-3 section 5.4.2
The brake system monitoring item consists of the following elements:
\begin{itemize}
\item Brake pedal position sensor
\item Brake light
\item Computing system.
\end{itemize}
The brake pedal position sensor must output the brake pedal position to the computing system.
The computing system must calculate the target brake light state and output it to the brake light.

The brake pedal position sensor must output the brake pedal position to the tractive control item, the high voltage control item, and the regenerative braking system.
