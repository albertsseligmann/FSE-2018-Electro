\section{Hazard analysis and risk assessment}
% Reference: DS/ISO 26262-3 section 7.4.1.1 -- 7.4.4.2
%
% Pre-requisites:
%    - Item definition
%
% The item is evaluated based ONLY on the item definition, i.e. WITHOUT considering
% internal safety mechanisms that are already implemented or are planned.
% Safety mechanisms that are intended to be implemented are instead part of the
% functional safety concept.
% 
% Reference: DS/ISO 26262-3 section 7.4.1

\subsection{Pre-existing information}
% This section is for further supporting information such as
%    - Change impact analysis if applicable (only for existing items being changed)
%    - Relevant information on other independent items
%
% Reference: DS/ISO 26262-3 section 7.3.2

\subsection{Situation analysis}
% The operational situations and operating modes in which an item's malfunctioning
% behaviour will result in a hazardous event are described.
% Consider both situations when the vehicle is correctly used AND when it is
% incorrectly used in a foreseeable manner.
% It is NOT necessary to consider situations in which the item is not expected to
% behave safely, i.e. when driving off-road.
%
% Required information:
%   - List of situations/modes in which the item may malfunction in a hazardous way
%
% Reference: DS/ISO 26262-3 section 7.4.2.1

\subsection{Hazard identification}
% Identify hazards systematically using e.g. brainstorming, FMEA and tests.
%
% A hazard is a potential source of harm caused by malfunctioning behaviour.
% A hazard is defined in terms of conditions or behaviour that can be observed at
% the vehicle level assuming every other independent item/system is functioning
% correctly.
% A hazard can have many potential causes related to the implementation of the item,
% but these do NOT need to be considered in this hazard analysis and risk assessment,
% which is derived from a functional behaviour of the item.

% A hazardous event is a combination of a hazard and an operational situation.
% A hazardous event may have several simultaneous consequences, which must be
% considered together.
%
% Required information:
%    - List of hazards that can be observed at the vehicle level
%    - List of hazardous events for combinations of hazards and operational situations
%    - The consequences of each hazardous event
%
% Reference: DS/ISO 26262-3 section 7.4.2.2
% Definitions: DS/ISO 26262-1 term 1.57 -- 1.59

\subsection{Classification of hazardous events}
% All hazardous events shall be classified, unless they are outside the scope of
% DS/ISO 26262, e.g.  electric shock or fires not directly caused by malfunctioning
% E/E systems.
%
% If there is doubt about the classification of a hazard, it should be classified 
% conservatively, i.e. a higher ASIL classification shall be given rather than a lower.
%
% Required information for each hazardous event:
%    - Estimate of severity class of potential harm from the event
%    - Probability class of exposure of each operational situation causing the event
%    - Estimate of controllability class of the event
%    - An ASIL for the event using the above parameters, ensuring that the operational
%      situations are not too granular so the exposure classes are inappropriately low
%
% References:
%	DS/ISO 26262-3 section 7.4.3 -- 7.4.4.2
%    - Severity class: DS/ISO 26262-3 Table 1
%    - Exposure probability class: DS/ISO 26262-3 Table 2
%    - Controllability class: DS/ISO 26262-3 Table 3
%    - ASIL determination: DS/ISO 26262-3 Table 4