\section{Software architectural design specification}
% Reference: DS/ISO 26262-6 section 7.1 -- 7.4.17
%
% Pre-requisites:
%    - Software safety requirements specification
%
% Reference: DS/ISO 26262-6 section 7.3.1

\subsection{Pre-existing information}
% The following information can be considered:
%    - Technical safety concept
%    - System design specification
%    - Available third party software components
%    - Tool application guidelines
%
% Reference: DS/ISO 26262-6 section 7.3.2

\subsection{SYSTEM NAME}
% The software architectural design represents all software components and their
% interactions in a hierarchical structure.
% Both static aspects, such as interfaces and data paths, and dynamic aspects, such
% as process sequences and timings, must be described.
% The software architectural design incorporates both safety and non-safety-related
% requirements, and should both implement the software safety requirements and
% provide a means to manage the complexity of the software components.
%
% Depending on the assigned ASIL of each software safety requirement, different
% notations should be used for the software architectural design.
%
% When designing the software architecture, the following shall be considered:
%    - Verifiability of the design (traceability between design and requirements)
%    - Suitability of configurable software
%    - Feasibility of design and implementation of the software units
%    - Testability of the architecture during integration testing
%    - Maintainability of the architecture
%
% To avoid failures resulting from high complexity software, the architectural design
% shall be modular, encapsulated and simple.
% These properties shall be obtained using the following principles
% (depending on the assigned ASIL):
%    - Hierarchical structure of software components
%    - Restricted size of software components
%    - Restricted size of interfaces
%    - High cohesion within each software component (separation of concerns)
%    - Restricted coupling between software components
%    - Appropriate scheduling properties
%    - Restricted use of interrupts/appropriate interrupt priorities
%
% The software architectural design shall be developed to the level where all software
% units are identified.
% The software architectural design shall describe:
%    - Software structure and hierarchical levels
%    - Logical sequence of data processing
%    - Data types and their characteristics
%    - External interfaces of the software components
%    - External interfaces of the software
%    - Constraints and external dependencies
%    - Functionality and behaviour
%    - Control flow and concurrency of processes
%    - Data flow between the software components
%    - Data flow at external interfaces
%    - Temporal constraints
%
% If a software components implements several software safety requirements, it shall
% be developed in compliance with the highest ASIL assigned to the safety requirements.
%
% For each software component assigned a software safety requirement, it shall be 
% evaluated whether any of the following error detection mechanisms should be applied,
% depending on the assigned ASIL of the software component:
%    - Range checks of input and output data
%    - Plausibility checks (e.g. referencing a model, comparing signals from different sources)
%    - Detection of data errors (e.g. error detecting codes, duplicate data storage)
%    - External monitoring (e.g. by an ASIC or external watchdog)
%    - Control flow monitoring
%    - Diverse software design
% Additionally, the following error handling mechanisms shall be applied, depending on
% the assigned ASIL of the software component:
%    - Static recovery (e.g. saving state before attempting an operation and reloading on failure)
%    - Graceful degradation
%    - Independent parallel redundancy (e.g. different algorithms in each parallel path)
%    - Error correcting codes for data
%
% An upper estimation of required resources for the software design shall be made, covering:
%    - Execution time
%    - Storage space
%    - Communication resources
%
% Reference:
%    - DS/ISO 26262-6 section 7.2 and 7.4.1 -- 7.4.17
%    - Software architectural design notations: DS/ISO 26262-6 Table 2
%    - Software architectural design principles: DS/ISO 26262-6 Table 3
%    - Error detection mechanisms: DS/ISO 26262-6 Table 4
%    - Error handling mechanisms: DS/ISO 26262-6 Table 5
% <<< Repeat this section for each system required to implement the item >>>