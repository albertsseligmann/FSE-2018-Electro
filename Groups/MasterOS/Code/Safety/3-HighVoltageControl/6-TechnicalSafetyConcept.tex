\section{Technical safety concept and system design specification}
% Reference: DS/ISO 26262-4 section 7
%
% Pre-requisites:
%    - Technical safety requirements specification
% 
% Reference: DS/ISO 26262-4 section 7.3.1

\subsection{Pre-existing information}
% The following information can be considered:
%    - Prior architectural assumptions
%    - Functional concept (non-safety related)
%    - Functional safety concept
%
% Reference: DS/ISO 26262-4 section 7.3.2

\subsection{SYSTEM NAME}
% The technical safety concept is the allocation of the technical safety requirements
% to system elements that are to be implemented by the system design.
% Each element shall inherit the highest ASIL form the technical safety requirements
% that it implements.
%
% The system architectural design implements the functional safety requirements,
% technical safety requirements and non-safety-related requirements.
% When implementing the technical safety requirements, the following shall be 
% considered:
%    - The ability to verify that the system design fulfills the requirements
%    - The technical capability of the design to achieve functional safety
%    - The ability to execute tests during system integration
%    - The allocation and partitioning decisions from the technical safety concept
%
% Required information:
%    - Allocation of technical safety requirements to hardware and/or software elements
%    - Interfaces between system elements and to other systems
%    - Measures for detection, control and mitigation of random hardware failures
%    - Qualitative safety analyses on the system design to identify and eliminate
%      systematic failures and faults
%
% Reference: DS/ISO 26262-4 section 7.4.1 -- 7.4.5
% <<< Repeat this section for each system required to implement the item >>>